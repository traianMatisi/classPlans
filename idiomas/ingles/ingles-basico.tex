\documentclass[12pt,a4paper]{article} %tipo de documento papel e tamanho da fonte
\usepackage[utf8x]{inputenc} %acentuação
\usepackage{ucs}
\usepackage[portuguese]{babel}
\usepackage[T1]{fontenc} %ajusta textos copiados/colados
\usepackage{amsmath} %símbolos matemáticos
\usepackage{amsfonts} %símbolos matemáticos
\usepackage{amssymb} %símbolos matemáticos
\usepackage{amsthm} %símbolos matemáticos
\usepackage{mathtools} %símbolos matemáticos
\usepackage{dsfont} %símbolos matemáticos
\usepackage{float}
\usepackage{makeidx}
\usepackage{graphicx} %permite inserir figuras
\usepackage{lmodern}
\usepackage{fourier}
\usepackage[left=2cm,right=2cm,top=2cm,bottom=2cm]{geometry} %layout da página
\usepackage{textcomp}
\usepackage{tabto} %permitir tabulação
\pagestyle{empty} %turn off page numbers (currently not working in my TeX distro)
\parindent 0px %turn off indentation
\title{English classes}
\author{Traian Matisi}

\begin{document}
\maketitle
\section{Class ONE}
\subsection{História da língua inglesa}

Tem raizes germânicas, devido a migração desses povos no século V d.C.\\
Os Anglo-saxões (tribos germânicas) conquistaram a grã-bretanha e o idioma se misturou com o existente.\\
Grande influência do Latim do Império Romano (século V a.C.).\\
Influência Celta (Bretões, povo original das Ilhas Britânicas).\\
Os países que atualmente têm o inglês como idioma oficial são:

\begin{itemize}
\item Europa:\\
Inglaterra\\
País de Gales\\
Escócia\\
Irlanda do Norte

\item América do Norte:\\
E.U.A.\\
Canadá

\item América Central:\\
Anguila\\
Antígua e Barbuda\\
Bahamas\\
Barbados\\
Belize\\
Bermudas\\
Ilhas Caiman\\
República Dominicana\\
Granada\\
Jamaica\\
Porto Rico\\
Santa Lúcia\\
São Vicente e Granadinas\\
Trinidade e Tobago\\
Turcos e Caicos\\
Ilhas Virgens Britânicas\\
Ilhas Virgens

\item América do Sul:\\
Guiana\\
Ilhas Malvinas

\item Ásia:\\
Índia\\
Filipinas\\
Cingapura

\item Oceania:\\
Austrália\\
Ilhas Fiji\\
Ilhas Maurício\\
Nauru\\
Nova Zelândia\\
Papua-Nova Guiné\\
Ilhas Salomão\\
Samoa Ocidental

\item África:\\
Botsuana\\
Camarões\\
Gâmbia\\
Gana\\
Quênia\\
Lesoto\\
Libéria\\
Maurício\\
Namíbia\\
Nigéria\\
Ruanda\\
Seicheles\\
África do Sul\\
Suazilândia\\
Tanzânia\\
Uganda\\
Zambia\\
Zimbábue
\end{itemize}
\subsection{Como estudar uma língua}

Gramática (duolingo)\\
Vocabulário (memrise)\\
Pronúncia (duolingo e memrise)\\
Expressões, jargões e gírias (ler)\\

\subsection{Hierarquia dos estudos de idiomas}

Fonética (th, r, i, ee, oo)\\
Fonologia (Lead vs Lead)\\
Ortografia (bus, bus', buses)\\
Morfologia (eat/ate, did/do)\\
Sintaxe (funções)\\
Semântica (significados)\\
Pragmática (utilidades)\\

\subsection{Dança da pronúncia}

Achieve vs Archive\\
Read vs Head\\
Lead vs Lead vs Led\\
By vs Bi vs Buy vs Bye\\
Two vs Chew\\
Data vs Data\\
Dinner vs Diner\\
Desert vs Dessert\\
Wave vs Waive\\
Peak vs Peek\\

Fair = fare\\
Find = fined\\
Flea = flee\\
Flew = flu\\
Flour = flower\\
For = four\\
Guessed = guest\\
Heal = heel\\
Hear = here\\
Hi = high\\
Higher = hire\\
Hour = our\\
Idle = idol\\
Knew = new\\
Knight = night\\
Know = no\\
Lean = lien\\
Made = maid\\
Mail = male\\
Meat = meet\\
Missed = mist\\
Pair = pear\\
Peace = piece\\
Plain = plane\\
Principal = principle\\
Read = red\\
Right = write\\
Cereal = serial\\
Cite = sight, site\\
Complement = compliment\\
Die = dye\\
Sale = sail\\
Scene = seen\\
Sea = see\\
Some = sum\\
son = sun\\
stair = stare\\
stake = steak\\
stationary = stationery\\
steal = steel\\
tale = tail\\
there = their, they're\\
threw = through\\
waist = waste\\
wait = weight\\
war = wore\\
wear = where\\
weak = week\\
weather = whether\\

Yes, english is hard, weird and can be very difficult.\\
But it can be understood, though, through tough, thorough thought.\\

\subsection{Estrangeirismos}

Basquete from basketball\\
Bife from beef\\
Blecaute from blackout\\
Coquetel from cocktail\\
Caubói from cowboy\\
Detetive from detective\\
Drinque from drink\\
Futebol from football\\
Gol from goal\\
Pênalti from penalty\\
Nocaute from knockout\\
Estresse from stress\\

Tecnologias\\

Backup\\
Chip\\
Download\\
Drive\\
Email/e-mail\\
Enter\\
Homepage/home page\\
Input\\
Internet\\
Mouse\\
Offline\\
Online\\
Outdoor\\
Output\\
Scanner\\ 
Site\\
Software\\
Stereo\\
Tablet\\
Upgrade\\
Upload\\
Website\\
Wireless\\

Ofícios\\

Feedback\\
Follow-up\\
Franchising\\
Freelance\\
Full-time\\
Holding\\
Know-how\\
Layout\\
Link\\
Lobby\\
Marketing\\
Meeting\\
Merchandising\\
Part-time\\
Performance\\
Press release\\
Ranking\\
Royalty\\
Slogan\\
Social media\\
Timing\\

Comidas\\

Bacon\\
Brunch\\
Cheeseburger\\
Cookie\\
Cupcake\\
Delivery\\
Diet\\
Fast-food\\
Freezer\\
Grill\\
Happy hour\\
Hot dog\\
Ketchup\\
Light\\
Milkshake/milk shake\\
Pub\\
Self-service\\
Snack bar\\
Sundae\\
Waffle\\

Esportes\\

Bike\\
Bodyboarding\\
Doping\\
Fitness\\
Jogging\\
Kart\\
Motocross\\
Mountain bike\\
Personal trainer\\
Skate \\
Windsurf\\

Moda\\

Baby doll\\
Black tie/black-tie\\
Blazer\\
Fashion\\
Jeans\\
Lycra\\
Shorts\\
Smoking\\
Stretch\\
Top model\\

Entretenimento\\

Best-seller\\
Camping\\
Hit\\
Hobby\\
Jazz\\
Jingle\\
Medley\\
Play\\
Script\\
Show\\
Spoiler\\
Stop\\
Thriller\\
Trailer\\
Videogame/video game\\
Rap\\
Remix\\
Rock\\

Outras\\

Airbag\\
Babysitter\\
Band-aid\\
Botox\\
Bullying\\
Cameraman\\
Check-in\\
Check-out\\
Check-up\\
Design\\
Display\\
Drive-thru\\
Flash\\
Flat\\
Freezer\\
Kit\\
Laser\\
Overdose\\
Shopping center\\
Slogan\\
Spray\\
Walkie-talkie\\
Workshop\\

\subsection{Dança da conjugação}

Verb to be\\

I.......am\\

You.....are\\

He......is\\
She.....is\\
It......is\\

We......are\\

You.....are\\

They....are\\

Verb to do\\

I.......do\\

You.....do\\

He......does\\
She.....does\\
It......does\\

We......do\\

You.....do\\

They....do\\

Verb will\\

I.......will\\

You.....will\\

He......will\\
She.....will\\
It......will\\

We......will\\

You.....will\\

They....will\\




\subsection{As classes de palavras}

Verbos (verbs)\\
Substantivos (nouns)\\
Adjetivos (adjectives)\\
Pronomes (pronouns)\\
Advérbios (adverbs)\\
Artigos (articles)\\
Preposições (prepositions)\\
Conjunções (conjunctions)\\
Numerais (numerals)\\
Interjeição (interjections)\\
Determinantes N/A (Determiners)\\

\subsection{Fixos (pre, in, suf)}

\subsection{Contrações}

\subsection{Vocabulário aula 1}

Pronomes pessoais:\\

I\\
You\\
He/She/It\\
We\\
You\\
They\\

Verbos:\\

to be\\
to have\\
to do\\
to go\\
to say\\
to think\\
to get\\
to let\\
to know\\
to see\\
to make\\
to want\\
to take\\
to come\\
to look\\
to find\\

\subsection{Common Nouns}

A noun is a word used to identify any of a class of people, places, or things (common noun),
or to name a particular one of these (proper noun).\\

-You can buy coffee at Starbucks.\\


A common noun is a noun denoting a class of objects or a concept as opposed to a particular individual.\\

-There was a sofa, two chairs, and a wardrobe in the room.\\

Note that common nouns are general names, they are not capitalized unless they begin a sentence or are part of a title.\\

-Capitals of the countries are usually very large cities. London is the capital of Great Britain.\\

Most of the time, we add -S to singular nouns to indicate plurality.\\

flower - flowers\\
dog - dogs\\

If the singular noun ends in -s, -ss, -sh, -ch, -x, -z, -o, add -es to make it plural\\

bus - buses\\
watch - watches\\
box - boxes\\
potato - potatoes\\

If the singular noun ends in -y, change to -i and add -es to make it plural, therefore adding -ies\\

baby - babies\\

If the singular noun ends in -f or -fe, -f is often changed into -ve before adding -s to make it plural\\

life - lives\\
wolf - wolves\\
belief - beliefs\\
chef - chefs\\

Some nouns do not follow any of the rules explained earlier. They are irregular. Here are the most common irregular nouns.\\

man - men\\
woman - women\\
person - people\\
child - children\\
tooth - teeth\\
foot - feet\\
mouse - mice\\

\subsection{Short Story Dialogue}

- I'm going grocery	shopping in a bit. Could I get you anything?\\
- I think we're running out of milk. You should buy that. And I don't mind some	cookies or candies.\\
- OK, I'll put it on my	list. Anything else?\\
- You can look up in the fridge	and just buy whatever you feel like we need. Oh, and don't forget to grab the newspaper	on your way back home. I'd really appreciate that!\\
- Roger! (=OK!/Understood!)\\

\subsection{Exercises}

Independent Practice - Common nouns\\
1 Transform the following singular nouns into plurals.\\
a. light –\\
b. man –\\
c. life – \\
d. lady –\\
e. tax – \\

2 Find mistakes in the following sentences.\\
a. Don't forget to take your Jacket! It's really cold outside today.\\
b. Your foot are really cold! You are freezing!\\
c. Elizabeth is a Doctor in a local hospital.\\
d. I like high waisted jeanses a lot. I feel really stylish wearing them.\\
e. There are many thiefes in this area. Be careful!\\

Independent Practice: Answers - Common nouns\\
1 Transform the following singular nouns into plurals.\\
a. light – lights\\
b. man – men\\
c. life – lives\\
d. lady – ladies\\
e. tax – taxes\\

2 Find mistakes in the following sentences.\\
a. Don't forget to take your jacket! It's really cold outside today.\\
b. Your foot/feet are really cold! You're freezing!\\
c. Elizabeth is a Doctor/doctor in a local hospital.\\
d. I like high waisted jeanses/jeans a lot. I feel really stylish wearing them.\\
e. There are many thiefes/thieves in this area. Be careful!\\
\section{Class TWO}
O idioma é uma ponte entre eu e os outros -Bakhtin\\
Mais importante ainda, é a lingua mais difundida internacionalmente devido à internet\\
Linguagem verbal - Escrita\\
Linguagem não verbal - Cores, desenhos\\
Linguagem híbrida - Quadrinhos\\
Pronúncia do alfabeto\\
Pronúncia dos sinais (vírgulas, pontos, exclamações, etc)

\subsection{Pronúncia da norma culta}
O alfabeto\\
As contrações\\
Sons mutos

\subsection{Pronúncia coloquial}
Cidade\\
Região\\
Faixa etária\\
Época\\
Tribos

\subsection{Pronúncias alternativas}
Contexto social\\
Contexto cultural\\
Contexto econômico

Mas professor, porquê aprender inglês?\\
Amplamente difundida hoje em dia\\
Oportunidades de emprego\\
Aproveitamento do conhecimento\\
Internet

Palavras de exemplos\\
A = ei\\
B = bi\\
C = ci\\
D = di\\
E = i\\
F = éf\\
G = dji\\
H = eidj\\
I = ái\\
J = djei\\
K = kei\\
L = él\\
M = em\\
N = en\\
O = ou\\
P = pi\\
Q = quiu\\
R = arr\\
S = éss\\
T = ti\\
U = iu\\
V = vi\\
W = dâbliu\\
X = eks\\
Y = uaí\\
Z = zi

Expressões comuns\\
Hello\\
Hi\\
Yo\\
Good morning\\
Good afternoon\\
Good evening\\
Good night\\
Excuse me\\
Beg your pardon\\
Come again\\
But\\
Altough\\
See you soon\\
See you later\\
See ya\\
Very well\\
Nice\\
Thank you varey much\\
Thank you\\
Thanks\\
ty\\
Please\\
I am sorry\\
I'm sorry\\
Yes\\
Yeah\\
Yep\\
Not (modifies words)\\
No (answer yes or no)\\
Nope\\
Nah\\
Bye\\
Bye bye\\
Goodbye

Criando diálogos\\

Diferenças entre o inglês americano e britânico\\

Pronomes (Pronouns)\\
Os Pronomes são classe de palavras que acompanham ou substituem um substantivo ou um outro pronome. Eles 	podem ser classificados como:\\
Pessoal – personal\\
Possessivo – possessive\\
Demonstrativo – demonstrative\\
Interrogativo - interrogative\\
Reflexivo – reflexive\\
Indefinido – indefinite\\
Relativo – relative\\
Recíproco - reciprocal\\

Para facilitar seu aprendizado, abordaremos neste curso os seguintes pronomes:\\
pessoal;\\
possessivo;\\
demonstrativo;\\
interrogativo.\\
Pronome Pessoal (Personal Pronoun)\\
Os Pronomes Pessoais referem-se a alguma pessoa, lugar ou objeto específico\\
e são subdivididos em:\\
Pronomes Pessoais do Caso Reto.\\
Pronomes Pessoais do Caso Oblíquo\\
\subsection{Adjectives}


\section{Class three}
\subsection{Greetings (Cumprimentos)}
\begin{itemize}
\item Hello!
\item Hi!
\item Yo!
\item What's up?
\item 'sup?
\item Good morning
\item Good afternoon
\item Good evening
\item Welcome!
\end{itemize}
\subsection{Farewells (Despedidas)}
\begin{itemize}
\item Bye!
\item Bye, bye!
\item See you tomorrow!
\item See you later!
\item See you soon!
\item See'ya!
\item Good night!
\item Have a nice day!
\item Have a nice weekend!
\item So long!
\end{itemize}
\subsection{Politeness (Etiqueta)}
\begin{itemize}
\item Please.
\item I'm <so> sorry.
\item My bad/fault/mistake.
\item Excuse me.
\item My apologies.
\item My condolences.
\item Thank you; Thank you, very much; Thanks; Thanks a lot; tnx, ty.
\item You're welcome; you are welcome.
\item Maybe.
\item Beg your pardon.
\item Come again?
\item Congratulations.
\item May I?
\item Sir/Mister/Miss/Missus/Lady/Milady/Doctor
\end{itemize}
\subsection{How to get by; to make do; to make ends meet (Se virando no 30)}
Visite o site www.teclasap.com.br\\
\begin{itemize}
\item I don't speak english very well... yet. I'm a quick study and learning fast.
\item Can you help me?
\item Where can i get some information about?
\item How much is this?
\item How do i get there?
\item Where can I get a cab/taxi?
\item Where is the subway station?
\item Where does it go?
\item May I use the restroom/toilet/bathroom/water closet?
\item Can I use the restroom/toilet/bathroom/water closet?
\item Don't mind if I do.
\item Nice to meet you.
\item How are you?
\item How are you doing?
\item Yes, yeah, yep.
\item No, nah, nope.
\item Not.
\item But.
\item Altough.
\item This.
\item These.
\item Those.
\item That.
\item Much.
\item Many.
\item A lot.
\item Lots <of>.
\item 
\item Pretty
\item Wonderfull.
\item Nice.
\end{itemize}
\begin{itemize}
\item Saudações de chegada\\
Hello! 	Olá!; Oi!\\
Hi! 	Olá!; Oi!\\
Hi, what's your name? 	Oi! Qual é o seu nome?\\
Good morning! 	Bom dia!\\
Good afternoon! 	Boa tarde!\\
Good evening! 	Boa noite!\\
Welcome! 	Seja bem-vindo(a)!
\item Saudações durante uma conversa\\
How are you? 	Como vai?\\
How are you doing? 	Como vai?\\
How is it going? 	Como vão as coisas?\\
How have you been? 	Como você tem estado/passado?\\
What's going on? 	O que está acontecendo?\\
What's up? 	E aí? O que você me conta?\\
What's new? 	O que você conta de novo?; Quais são as novidades?\\
What have you been up to all these years? 	Onde você esteve esses anos?\\
Where have you been hiding? 	Por onde você andava?\\
It's been a long time! 	Quanto tempo!\\
It’s been ages since I’ve seen you. 	Faz um tempão que eu não te vejo.\\
How long has it been? 	Quanto tempo!\\
Long time no see! 	Há quanto tempo não te vejo!\\
It's been too long! 	Você sumiu!\\
Glad to meet you! 	Prazer em conhecê-lo!\\
Nice to meet you! 	Prazer em conhecê-lo!\\
Nice to meet you too! 	Prazer em conhecer você também!\\
Pleased to meet you! 	Prazer em conhecê-lo!\\
The pleasure is mine! 	O prazer é meu!\\
My pleasure! 	O prazer é meu!\\
It’s always a pleasure to see you. 	É sempre um prazer te ver!
\item Saudações de despedida\\
Good night! 	Boa noite!\\
See you later! 	Até logo!\\
See ya! 	Até logo!\\
Until next time! 	Até a próxima!\\
See you tomorrow! 	Até amanhã!\\
Goodbye! 	Tchau!\\
Bye! 	Tchau!\\
Bye-bye 	Tchau!\\
Pleased to meet you! 	Prazer em conhecê-lo!\\
Nice to meet you! 	Prazer em conhecê-lo!\\
Take care! 	Se cuida!\\
Have a nice weekend! 	Bom fim de semana!\\
Have a nice day! 	Tenha um bom dia!\\
So long! 	Até!
\item Saudações formais iniciais para cartas e e-mails\\
Dear Sir, 	Prezado/Caro/Exmo. Senhor,\\
Dear Madam, 	Prezada/Cara/Exma. Senhora,\\
Dear Mr. (+sobrenome), 	Prezado/Caro Senhor (+sobrenome),\\
Dear Mrs. (+sobrenome), 	Prezada/Cara Senhora (+sobrenome),\\
\item Saudações formais finais para cartas e e-mails\\
Sincerely, 	Atenciosamente,; Atentamente,; Cordialmente,\\
Sincerely yours, 	Atenciosamente,; Atentamente,; Cordialmente,\\
Yours sincerely, 	Atenciosamente,; Atentamente,; Cordialmente,\\
Yours faithfully, 	Atenciosamente,; Atentamente,; Cordialmente,\\
Yours truly, 	Atenciosamente,; Atentamente,; Cordialmente,
\item Saudações informais iniciais para cartas e e-mails
Dear ..., 	Querido(a)...,\\
Dear friend, 	Querido(a) amigo(a),\\
Hi..., 	Oi...,; Olá...,\\
Hello ..., 	Oi...,; Olá...,\\
My dear ..., 	Meu querido...,/Minha querida...,
\item Saudações informais finais para cartas e e-mails\\
Inglês 	Português\\
Cheers, 	Um abraço,\\
See you, 	Até logo,\\
With love, 	Com afeto,\\
Love, 	Com afeto,\\
Best wishes, 	Um abraço,\\
Kisses, 	Beijos,\\
Kisses and hugs, 	Beijos e abraços\\
XOXO, 	Beijos e abraços\\
Regards, 	Saudações,\\
Kind regards, 	Saudações,\\
Best regards, 	Saudações,\\
\end{itemize}
\section{Class four}
\subsection{Questioning and asking}
\begin{itemize}
\item You do / Do you do?
\item I am / Am I?
\item You are / Are you?
\end{itemize}
\subsection{Modal verbs}
Importante: Quando se usa o verbo auxiliar, não se conjuga o verbo principal.\\
Possuem função social, dá idéia de: Se algo é possível, necessário, permitido, proibido,habilidade, conselho, sugestão, etc
\begin{itemize}
\item Do - Fazer.
\item Can - Poder.
\item Have - Tenho.
\item Could - Poderia.
\item Should - Deveria.
\item Shall - Hei.
\item Will - Irá* (tradução adaptada, o verbo will apenas tramsforma o verbo principal em futuro e não tem tradução).
\item Would - Iria* (tradução adaptada também).
\item May - Dever/Poder.
\item Might - Dever/Talvez (maybe apenas sem verbos e perhaps).
\item Must - Dever
\item Ought - Dever
\end{itemize}
\subsection{Negatives}
Para negar:
\begin{enumerate}
\item Verbo \textbf{TO BE} seguido de \textbf{NOT}.
\item No caso da ausência do verbo \textbf{TO BE} por qualquer motivo, adicionamos um \textbf{verbo modal} e o negamos.
\item Verbo MODAL seguido de not (e pode ser seguido ou não do verbo principal, o qual não é conjugado pois há presença do auxiliar/modal).
\end{enumerate}
\begin{itemize}
\item No: Opposes the yes. Can oppose something, meaning zero something.
\item Not: Modifies the verb. Works with the verb or noun.
\item Do not do / Does not do: Modifies the sentence. Works with all the auxiliary verb.
\item Do do / Does do
\end{itemize}
\textbf{O \textit{verbo to be} ficaria assim:}
\begin{itemize}
\item I am not / I'm not / n/a
\item You are not / You're not / You aren't
\item He is not / He's not / He isn't
\end{itemize}
\textbf{Os \textit{verbos modais} ficariam assim:}
\begin{itemize}
\item Do not - Don't
\item Can not* - Cannot - Can't
\item Have not - Haven't
\item Could not - Couldn't
\item Should not - Shouldn't
\item Shall not - Shan't*
\item Will not - Won't
\item Would not - Wouldn't
\item May not
\item Might not
\item Must not - Mustn't
\item Ought not
\end{itemize}
\section{Class five}
\subsection{Indicatives}
\begin{itemize}
\item What:\\ What's your name?
\item How:\\ How are you?
\item When:\\ When is your birthday?
\item Who:\\Who are you?
\item Why:\\ Porquês.\\
Por que - Pergunta de "Por qual motivo/Pelo qual".\\
Porque - Resposta ou conjunção.\\
Por quê - Final de frases que não sejam perguntas.\\
Porquê - Because - Substantivo motivo/razão.
\item This:\\ Esse/este/isso/isto
\item That:\\ Aquele
\item These:\\ Esses/estes
\item Those:\\ Aqueles
\item Much:\\ Quanto/Tanto
\item Many:\\ Quanto/Vários/Muitos
\item Lot:\\ Muito
\item Very:\\ Thank you very much.
\item Few:\\ Pouco.
\item Little:\\ Pequeno/Pouco.
\item Big
\item Tall
\item Short
\item Large
\item Have
\item Can
\item May
\item Any
\item All
\item Bring
\item Take
\item Get
\item Nice
\item Well
\item Bad
\item Good
\item Evil
\item In
\item Inside
\item Out
\item Outside
\item On
\item On top
\item Over
\item Under
\item Above
\item Below
\item In front
\item Behind
\item On the left
\item On the right
\item Between
\item Among
\item Against
\item On the floor
\end{itemize}
\section{Class SIX}
\subsection{Object pronouns}
Sempre depois do verbo ou de preposição.
\begin{itemize}
\item I/Me
\item You/You
\item He/Him
\item She/Her
\item It/It
\item We/Us
\item You/You
\item They/Them
\end{itemize}
Podemos usar como exemplo to love, to study and to speak
Exemplos
\begin{itemize}
\item I love me/you/him/her/it/us/you/them 
\item You love me/you/him/her/it/us/you/them
\item He loves me/you/him/her/it/us/you/them
\item She loves me/you/him/her/it/us/you/them
\item It loves me/you/him/her/it/us/you/them
\item We love me/you/him/her/it/us/you/them
\item You love me/you/him/her/it/us/you/them
\item They love me/you/him/her/it/us/you/them
\end{itemize}
As preposições
\begin{itemize}
\item I am/He is/They are in love \textbf{with} me/you/him/her/it/us/you/them
\item Front
\item Back
\item Next
\item Far
\item Aboard
\item About
\item Above
\item According
\item Across
\item After
\item Against
\item Among
\item Along
\item Around
\item At
\item Before
\item Below
\item Between
\item By
\item Down
\item During
\item For
\item From
\item In
\item Inside
\item Into
\item Like
\item Near
\item Of
\item Off
\item On
\item Out
\item Over
\item Since
\item Through
\item To
\item Towards
\item Until
\item Under
\item Up
\item With
\item Without
\end{itemize}
\subsection{Verb tenses - Tempos verbais}
\textit{Pode ser útil fazer o continuous antes de continuar}:
Vamos conjugar os verbos e depois vamos negar e perguntar também\\
Para lembrar de todos os tempos verbais, apenas precisamos lembrar de 5 palavras. Present, past, future, continuous, perfect.
\begin{enumerate}
\item Simple present - Presente simples (love/loves).
\item Present continuous/progressive (am/are/is loving).
\item Simple past - Passado simples (loved)
\item Past continuous/progressive (were/was loving)
\item Simple future - Futuro simples (will love)
\item Future continuous/progressive (will be loving)
\item Perfect present simple (have/has loved)
\item Perfect present continuous/progressive (have been loving)
\item Perfect past simple (had loved)
\item Perfect past continuous (had been loving)
\item Perfect future (will have loved)
\item Perfect future continuous/progressive (will have been loving)
\end{enumerate}
\end{document}