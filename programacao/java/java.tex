\documentclass[12pt,a4paper]{article} %tipo de documento papel e tamanho da fonte

\usepackage[utf8x]{inputenc} %acentuação
\usepackage{ucs}
\usepackage[portuguese]{babel}
\usepackage[T1]{fontenc} %ajusta textos copiados/colados
\usepackage{amsmath} %símbolos matemáticos
\usepackage{amsfonts} %símbolos matemáticos
\usepackage{amssymb} %símbolos matemáticos
\usepackage{amsthm} %símbolos matemáticos
\usepackage{mathtools} %símbolos matemáticos
\usepackage{dsfont} %símbolos matemáticos
\usepackage{float}
\usepackage{makeidx}
\usepackage{graphicx} %permite inserir figuras
\usepackage{lmodern}
\usepackage{fourier}
\usepackage[left=1cm,right=1cm,top=1cm,bottom=1cm]{geometry} %layout da página
\usepackage{textcomp}
\usepackage{tabto} %permitir tabulação
\usepackage{hyperref} % You can use \url with \usepackage{hyperref} \url{http://stackoverflow.com/}
\usepackage{todonotes} %enables you to insert small notes, like \todo{Rewrite this answer \ldots}

\pagestyle{empty} %turn off page numbers (currently not working in my TeX distro)
\parindent 0px %turn off indentation

\author{Traian Matisi}
\title{Java}
%date{}

\begin{document}
\maketitle

\section{Primeiro programa em Java}
\subsection{hello.java}
Tudo o q escrevermos aqui pode ser digitado no jshell, o qual é uado para testar snippets ou pequenos trechos de código ou mesmo blocos de código. Para mais referências consulte youtube.\\
Focaremos aqui no uso de IDEs (Netbeans ou IntelliJ)
\begin{enumerate}
\item package hellofriend;
\item 
\item public class HelloFriend \{
\item \tabto{1.1cm}public static void main(String[] args)\{
\item \tabto{2.2cm}System.out.println("Hello, friend!");
\item \tabto{1.1cm}\}
\item \}
\end{enumerate}
O programa acima imprime na tela a string literal (linha de texto colocada entre as aspas duplas) "Hello, friend".\\
Sobre os detalhes técnicos do código, os pacotes se escrevem com todas as letras minúsculas (e talvez inclua underline ou hífen), as interfaces e classes usam CamelCase com a primeira maiúscula. Atributos, métodos e variáveis usam camelCase com a primeira letra minúscula (veremos logo o que cada coisa é). Constantes são escritas em letras MAIÚSCULAS.

\subsection{keywords}
Em Java, as palavras reservadas são case sensitive. Logo \textit{\textbf{Int x}} (uma string literal) é diferente de \textit{\textbf{int x}} (uma variável inteira).\\
Variáveis são nome que damos a espaços na memória os quais usamos para armazenar e manipular dados. Existem diversos tipos de dados, podemos criar nossos próprios tipos, mas os tipos primitivo são palavras reservadas ou \textit{keywords} em Java. As linguagens de programação chamam a menor unidade de processo de \textbf{comando}, ou também \textbf{statement}. O conteúdo da variável do lado direito do sinal de atribuição ($=$) é chamado de expressão. Expressões podem ser também operações desde que o resultado delas seja um único resultado e não um intervalo.\\


\end{document}