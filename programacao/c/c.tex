\documentclass[12pt,a4paper]{article} %tipo de documento papel e tamanho da fonte

\usepackage[utf8x]{inputenc} %acentuação
\usepackage{ucs}
\usepackage[portuguese]{babel}
\usepackage[T1]{fontenc} %ajusta textos copiados/colados
\usepackage{amsmath} %símbolos matemáticos
\usepackage{amsfonts} %símbolos matemáticos
\usepackage{amssymb} %símbolos matemáticos
\usepackage{amsthm} %símbolos matemáticos
\usepackage{mathtools} %símbolos matemáticos
\usepackage{dsfont} %símbolos matemáticos
\usepackage{float}
\usepackage{makeidx}
\usepackage{graphicx} %permite inserir figuras
\usepackage{lmodern}
\usepackage{fourier}
\usepackage[left=1cm,right=1cm,top=1cm,bottom=1cm]{geometry} %layout da página
\usepackage{textcomp}
\usepackage{tabto} %permitir tabulação
\usepackage{hyperref} % You can use \url with \usepackage{hyperref} \url{http://stackoverflow.com/}
\usepackage{todonotes} %enables you to insert small notes, like \todo{Rewrite this answer \ldots}

\pagestyle{empty} %turn off page numbers (currently not working in my TeX distro)
\parindent 0px %turn off indentation

\author{Traian Matisi}
\title{Programação em C}
%date{}

\begin{document}
\maketitle

\section{Outputs - As saídas de dados}

\subsection{hello.c}
\begin{enumerate}
\item \#include <stdio.h>
\item 
\item int main(void)\{
\item \tabto{1.1cm} printf("Hello, friend!$\backslash$n");
\item \}
\end{enumerate}
\tabto{2cm}O código acima escreve no terminal a frase dentro dos parenteses uma única vez.
Para compilar e executar respectivamente o código usando GCC usamos os comandos
\begin{enumerate}
\item gcc <\emph{nome do código fonte}> -o <\emph{nome que escolhermos para o executável}>
\item ./<\emph{nome que escolhermos para o executável}>
\end{enumerate}

\subsection{Compilação em máquinas diferentes}
Uma compilação simples não levará mais argumentos do que indicados acima. Aconselho você a procurar outros argumentos úteis como inclusão de bibliotecas e flags de compilação para seu programa C de acordo com sua necessidade.
\begin{itemize}
\item No linux:\\gcc hello.c -o hello\\./hello
\item No windows:\\gcc hello.c -o hello.exe\\.$\backslash$hello.exe
\end{itemize}

\subsection{Elementos do printf()}
\begin{itemize}
\item necessário biblioteca \textbf{\# include <stdio.h>} para usarmos o printf.
\item Os argumentos do parenteses do printf("string q eu queira + máscaras de \%tipos", <variáveis>)
\item Usamos o caracter de escape $\backslash$ para tabulações e linhas novas.
\end{itemize}

\section{Variáveis - Caixas etiquetadas para tipos de dados}
\tabto{2cm}Variáveis são como caixas etiquetadas e dentro dessas caixas podemos guardar qualquer dado que quisermos, desde que respeitado o tipo de dados da variável, ou seja, cada caixa etiquetada guarda apenas um tipo de dado.

\subsection{Tipos de variáveis}
\begin{itemize}
\item Caracteres - char; \%c; \%s
\item Strings[n] - char; \%s
\item Números inteiros - int; \%i; \%d
\item Números com vírgula e 6 casas decimais - float; \%f
\item Números com vírgula e 12 casas decimais - double; \%lf
\item Valores booleanos - bool
\item Números naturais - unsigned
\item Números inteiros - signed
\item Números grandes - long
\item Números pequenos - short
\end{itemize}

\tabto{2cm}Variáveis podem ser modificadas durante a execução do programa.
\tabto{2cm}Constantes não podem ser modificadas durante a execução do programa. Teremos um erro de compilação.

\subsection{hello.c}
\tabto{2cm}Modificamos o primeiro programa pra usarmos uma variável para a string, e usamos o formatador para strings \%s
\begin{enumerate}
\item \# include <stdio.h>
\item 
\item int main(void)\{
\item \tabto{1.1cm} char palavra[] = "friend!";
\item \tabto{1.1cm} printf("Hello, \%s$\backslash$n", palavra);
\item \}
\end{enumerate}

\subsection{Bibliotecas math.h e cmath.h}
Some functions can be called inside other functions like the printf("\%f", pow(a, b))
\begin{itemize}
\item pow(x, y)
\item sqrt(x)
\item ceil(x)
\item floor(x)
\end{itemize}

\subsection{Comentários}
\begin{itemize}
\item /*\\multi\\line\\comments\\ */
\item //one-liner comment
\end{itemize}

\subsection{CONSTANTES}
\begin{itemize}
\item cont int NUM = 11; //dentro de main() ou outras funções e em caixa alta
\item CONST 
\item Conteúdo vindo do printf() também é considerado CONSTANTE
\end{itemize}

\section{Inputs - As entradas de dados}

\subsection{Elementos do scanf()}
Existem diversas maneiras de promptar o usuário, scanf é a mais simples, mas há também getchar, fgets, etc.\\
A função scanf() é a complementar da printf(), a sintaxe é como segue:\\
\textbf{scanf("\%i", \&num)} //Captura um número inteiro e guarda na variável \textit{num}. \& é um ponteiro para \textit{num}

Para strings é levemente diferente.\\
\begin{enumerate}
\item \#include <stdio.h> 
\item \#include <stdlib.h>
\item 
\item int main(void)\{
\item \tabto{1.1cm}char nome[25]; \textit{//"Aloca 24 bytes (letras) + $\backslash 0$ para uma string.}
\item \tabto{1.1cm}printf("Digite seu nome:$\backslash$t");
\item \tabto{1.1cm}scanf("\%s", nome); \textit{//Para strings não usamos \& para ponteiro.}
\item \tabto{1.1cm}printf("Muito prazer, \%s", nome); \textit{//Escreve no terminal o nome digitado.}
\item \tabto{1.1cm}return 0;
\item \}
\end{enumerate}

Porém, por razões históricas, se houver espaço entre as palavras, o scnaf pára de capturar o prompt. Para isso podemos usar uma função "melhor", fgets.

\begin{enumerate}
\item \#include <stdio.h> 
\item \#include <stdlib.h>
\item 
\item int main(void)\{
\item \tabto{1.1cm}char nome[25]; \textit{//"Aloca 24 bytes (letras) + $\backslash 0$ para uma string.}
\item \tabto{1.1cm}printf("Digite seu nome:$\backslash$t");
\item \tabto{1.1cm}fgets(nome, 25, stdin); \textit{//Alternativa para o problema do espaço do scanf}
\item \tabto{1.1cm}printf("Muito prazer, \%s", nome); \textit{//Escreve no terminal o nome digitado.}
\item \tabto{1.1cm}return 0;
\item \}
\end{enumerate}

\section{iv}
\section{v}
\section{vi}
\section{vii}
\section{viii}
\section{ix}
\section{x}
\section{xi}
\section{xii}
\section{xiii}
\section{xiv}
\section{xv}

\end{document}